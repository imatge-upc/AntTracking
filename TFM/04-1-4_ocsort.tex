
{
    \ac{OC-SORT}\cite{cao2023observation} is a 2021 model that builds upon the BYTE model. 
    It enhances the motion model, introduces an improved location score, and adds a third association stage, 
    excluding the appearance model for research purposes.
}

\paragraph{Motion model}

{
    In \ac{OC-SORT}, the Kalman filter is improved by freezing its state until a new observation is assigned. 
    The model updates each skipped frame using a linear model when new observations arrive (observation centric), 
    other models update their state with their own estimations (estimation centric). 
    \ac{SORT} is able to roughly follow a non-linear motion if the time between observations is short enough, 
    \ac{OC-SORT} reduce the estimation noise originated on missed frames.
} 

\paragraph{Location score}

{
    The location score from \ac{OC-SORT} is a global direction metric. 
    To compute this metric, \ac{OC-SORT} performs a search of the most relevant older observation:
    the oldest observation from the last $\Delta t=3$ frames, or the newest observation available.
    The relevant observation and the newest observation available are used to obtain the track global direction vector:
}

\begin{equation}
    \label{eqn:track direction}
    direction = \frac{(z_{new} - z_{old})}{||(z_{new} - z_{old})||}
\end{equation}

{
    A potential global direction is computed using the set of high score detections as potential new observations. 
    At the end, each track has its own global direction and the hypothetical global direction in the case of using a certain new detection.
}

%\begin{subequations}
%    \begin{align}
%        \mathbf{\Delta x_{tracks}} &= \begin{bmatrix}
%            direction_{1}[x] & direction_{1}[x] & \hdots & direction_{1}[x] \\
%            direction_{2}[x] & direction_{2}[x] & \hdots & direction_{2}[x] \\
%            \vdots & \vdots & \vdots & \vdots \\
%            direction_{M}[x] & direction_{M}[x] & \hdots & direction_{M}[x]
%            \end{bmatrix}
%        \label{eqn:dirction x tracks} \\[0.5cm]
%        \mathbf{\Delta y_{tracks}} &= \begin{bmatrix} \\
%            direction_{1}[y] & direction_{1}[y] & \hdots & direction_{1}[y] \\
%            direction_{2}[y] & direction_{2}[y] & \hdots & direction_{2}[y] \\
%            \vdots & \vdots & \vdots & \vdots \\
%            direction_{M}[y] & direction_{M}[y] & \hdots & direction_{M}[y]
%            \end{bmatrix}
%        \label{eqn:dirction y tracks} \\[0.5cm]
%        \mathbf{\Delta x_{detections}} &= \begin{bmatrix}
%            direction_{1,1}[x] & direction_{1,2}[x] & \hdots & direction_{1,N}[x] \\
%            direction_{2,1}[x] & direction_{2,2}[x] & \hdots & direction_{2,N}[x] \\
%            \vdots & \vdots & \vdots & \vdots \\
%            direction_{M,1}[x] & direction_{M,2}[x] & \hdots & direction_{M,N}[x]
%            \end{bmatrix}
%            \label{eqn:dirction x detections} \\[0.5cm]
%            \mathbf{\Delta y_{detections}} &= \begin{bmatrix}
%                direction_{1,1}[y] & direction_{1,2}[y] & \hdots & direction_{1,N}[y] \\
%                direction_{2,1}[y] & direction_{2,2}[y] & \hdots & direction_{2,N}[y] \\
%                \vdots & \vdots & \vdots & \vdots \\
%                direction_{M,1}[y] & direction_{M,2}[y] & \hdots & direction_{M,N}[y]
%            \end{bmatrix}
%            \label{eqn:dirction y detections}
%    \end{align}
%\end{subequations}

{
    These directions serve to obtain the angular distance $\mathbf{\Delta\phi}$ (from 0 to $\pi$) between the track and the detections trajectories, 
    later, it is transformed into a normalized angular score (from $-0.5$ to $0.5$) with the equation \ref{eqn:angular score}.
}


\begin{equation}
    \label{eqn:angular score}
%    \begin{split}
%        &\mathbf{\Delta\phi} = arccos(\Delta x_{tracks} \odot \Delta x_{detections} + \Delta y_{tracks} \odot \Delta y_{detections})\\
%        &\mathbf{\phi_{score}} = 0.5 - \left(\frac{|\mathbf{\Delta\phi}|}{\pi}\right)
%    \end{split}
    \mathbf{\phi_{score}} = 0.5 - \left(\frac{|\mathbf{\Delta\phi}|}{\pi}\right)
\end{equation}

\needspace{0.1\textheight}

\paragraph{Associator}

{
    During the association step, \ac{OC-SORT} categorizes the detections in three groups:
}

\begin{enumerate}
    \item Initially high score detections.
    \item Initially low score detections.
    \item High score associations that were not successfully associated
\end{enumerate}

{
    The first step performs a linear assignment with the Hungarian algorithm. 
    Using a weighted sum of the \ac{IoU} and the angular score from equation \ref{eqn:angular score} 
    pondered by the detections confidence ($c_{n}$) using the Hadamard product\footnote{The notation for element-wise product will be $\odot$ because $\oslash$ could be used for element-wise division.} as the cost:
}

\begin{equation}
    \label{eqn:detection confidence matrix}
    \mathbf{confidence_{detections}} = \begin{bmatrix}
        c_{1} & c_{2} & \hdots & c_{N} \\
        c_{1} & c_{2} & \hdots & c_{N} \\
        \vdots & \vdots & \vdots & \vdots \\ 
        c_{1} & c_{2} & \hdots & c_{N}
    \end{bmatrix}
\end{equation}

\begin{equation}
    \label{eqn:OC-SORT first association cost}
    cost_{1} = -\mathbf{IoU} - \lambda \cdot (\mathbf{\phi_{score}} \odot \mathbf{confidence_{detections}})
\end{equation}

{
    At the end of the first stage, a threshold on the \ac{IoU} is applied.
}

{
    The second association stage is the same as the \ac{BYTE} model (section \ref{sec:BYTE}): 
    the low confidence detections are assigned with a second location metric and threshold.
}
 
{
    The third stage of \ac{OC-SORT} is the same as the second stage; 
    however, it uses the last known observation of the remaining unmatched tracks and the remaining high confidence detections.
}

\subsubsection{Location metrics from OC-SORT}

{
    While not originally part of OC-SORT, the publicly available code from the OC-SORT authors includes a range of location metrics. 
    These metrics were initially developed for object detection but have found utility in object tracking scenarios, 
    particularly in improving the accuracy of associations compared to traditional \ac{IoU} by extending the overlapping ratio with a nearness ratio. 
    This section presents an exploration of three location metrics utilized in OC-SORT's codebase.
}

\paragraph{GIoU} 

{
    The \acl{GIoU}\cite{Rezatofighi_2018_CVPR} is a metric that extends the \ac{IoU} into the negatives and smoothen it by accounting for both positive and negative scenarios.
    However, it scores positively bigger boxes. 
    The upgrade consist in subtracting a term computed from the enclosure area and the complementary of the union with the enclosure area:
}


\begin{equation}
    \label{eqn:GIoU}
%    \begin{split}
%        GIoU(A, B) &= IoU(A, B) - \frac{|enclosure(A, B)| - |A \cup B|}{|enclosure(A, B)|} \\[0.25cm]
%        with \; |enclosure(A, B)| &= (max_{x2}(A, B) - min_{x1}(A, B)) \cdot (max_{y2}(A, B) - min_{y1}(A, B))
%    \end{split}
    GIoU(A, B) = IoU(A, B) - \frac{|enclosure(A, B)| - |A \cup B|}{|enclosure(A, B)|}
\end{equation}



\paragraph{DIoU} 

{
    The \acl{DIoU}\cite{Zheng_Wang_Liu_Li_Ye_Ren_2020} is a metric similar to the \ac{GIoU}, but with a different approach. 
    It replace the previous enclosure distance term with a the novel metric called \ac{IoO}. 
    \ac{IoO} is computed as the center distance normalized by the enclosure diagonal lenght:
}

\begin{equation}
    \label{eqn:DIoU}
    \begin{split}
        D&IoU(A, B) = IoU(A, B) - \frac{InnerDistance(A, B)}{OuterDistance(A, B)} \\[0.25cm]
        with& \; InnerDistance(A, B) = d(center(A), center(B)) \\[0.25cm]
        with& \; OuterDistance(A, B) = d(BottomRight(AB), UpperLeft(AB)) \\[0.25cm]
%        with& \; UpperLeft(AB) = [min_{x1}(A, B), min_{y1}(A, B)] \\[0.25cm]
%        with& \; BottomRight(AB) = [max_{x2}(A, B), max_{y2}(A, B)] \\[0.25cm]
%        with& \; d(P_{1}, P_{2}) = \sqrt{(P_{2_{x}} - P_{1_{x}})^{2} + (P_{2_{x}} - P_{1_{x}})^{2}}\\[0.25cm]
        &\text{using AB as a shortened notation for enclosure(A, B)}
    \end{split}
\end{equation}

\paragraph{CIoU}

{
    The \acl{CIoU}\cite{zheng2021ciou}, also from the same authors as \ac{DIoU}, further refines object association by accounting for variations in aspect ratios. 
    This new feature requires the computation of an aspect ratio distance, the authors uses a normalized quadratic angular distance, and an additional cost value based in the IoU. 
    The \ac{CIoU} score is defined as:
}

\begin{equation}
    \label{eqn:CIoU}
    \begin{split}
        C&IoU(A, B) = DIoU(A, B) - AspectRatioDistance(A, B) \cdot AspectRatioCost(A, B) \\[0.25cm]
        with& \; AspectRatioDistance(A, B) = \frac{4}{\pi^2} \cdot \left(arctan\left(\frac{B_{w}}{B_{h}}\right) - arctan\left(\frac{A_{w}}{A_{h}}\right)\right)^{2} \\[0.25cm]
        with& \; AspectRatioCost(A, B) = \frac{AspectRatioDistance(A, B)}{AspectRatioDistance(A, B) + 1 - IoU}
    \end{split}
\end{equation}
