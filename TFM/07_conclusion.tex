
\section{Conclusions}

{
    This work started an ant tracking project researching and summarizing the theoretical background. 
    It also researched the state of the art and found the need to upgrade the available ant tracking applications with the state of the art models.
}

{
    A starting project lacks in data, within the realization of this project new raw data was obtained and some of it was labeled, 
    tools to ease the labeling process were researched and developed. 
    From this data, a \ac{YOLOv8n} detector and a \ac{BoT} appearance model were trained.
}

{
    The \ac{YOLOv8n} detector with a \ac{mAP}@50-95 of 85.5\% on validation was trained 
    and applied on tracking models using a slicing windows with \ac{SAHI}. 
    However, in the testing sequence it reached a \ac{mAP}@50-95 of 15\%. 
}

{
    The \ac{BoT} appearance model was unsuccessful. 
    Although the training reached a 74\% of accuracy, 
    an analisis of the model outputs concluded hat the appearance features have to be further researched.
    Additionally, its integration within the Deep SORT model yielded catastrophic results with an association metric (\ac{AssA}) of 1\%.
}

{
    It was found that, in the current state of the research, the detection component is the most relevant one.
    Detection upgrades (\ac{DetA}) were strongly correladed with association improvement (\ac{AssA}) and, subsequently, with \ac{HOTA}.
}

{
    Finally, a set of trackers were tested setting a baseline of 49\% of \ac{HOTA} with the OC-SORT with YOLOv8n model for future projects in this ant tracking problem.
}
