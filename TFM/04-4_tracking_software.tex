
{
    Tracking ants and insects poses unique challenges and is not a straightforward task. 
    Moreover, trackers based on appearance models require species-specific tuning. 
    After extensive research on the topic, five applications and one research were found.
}

\subsubsection{AnimalTA}

{
    AnimalTA\cite{animalTA}, developed in 2023, is an application designed for animal tracking based on minimum distance with respect to the last observation. 
    However, it lacks motion estimation capabilities, making it older in terms of technology compared to SORT. 
    The detection process relies on background extraction, binarization, and connected component analysis. 
    It's worth noting that using connected components for detection can lead to challenges in crowded environments, as objects in close proximity may become fused. 
    Empirical observations suggest that AnimalTA's tracking performance is inferior to that of the SORT model.
}

{
    In this context, \textbf{background extraction} refers to a technique that models the background from a video to subtract it from each frame, isolating the foreground. 
    This is typically achieved by training a parametric model, such as a Gaussian Mixture Model, with the previous frames as training data. 
    All frames can be used for training since foreground objects are transient and do not affect the model's memory.
}

\subsubsection{AnTracks}

{
    AnTracks\cite{anTracks}, developed in 2010, is an application that accommodates various object detection tools, including background extraction. 
    Although the specifics of their tracking algorithm are not public, observed performance indicates it falls behind of AnimalTA.
}

\subsubsection{AnTraX}

    
{
    AnTraX\cite{anTraX} is an application from 2020. It uses a background extraction detector. 
    The tracking is performed in two stages: 
}

\begin{itemize}
    \item The first stage is based on optical flow, it only assign identities when there is a high level of certainty, making tracklets.
    \item The second stage join the tracklets into tracks using a small \ac{CNN}-based appearance model (tracklets that are simultaneous are discarded).
\end{itemize}

\subsubsection{idTracker}

{
    IdTracker\cite{IdTracker, IdTracker2}, initially developed in 2013 and upgraded in 2018 by a group of researchers from \ac{CSIC}, relies on background extraction for detection. 
    Similar to Deep SORT, it employs appearance descriptors obtained from a small \ac{CNN}. 
    However, a notable limitation is its requirement for prior knowledge of the number of tracked objects.
}

\subsubsection{Ctrax}

{
    Ctrax\cite{ctrax}, introduced in 2009, is noteworthy because its algorithm closely resembles \ac{SORT}\footnote{SORT is a model from the 2016.} (see Figure \ref{fig:sort}), although it has certain components downgraded:
}

\begin{enumerate}
    \item Ctrax \textbf{detection step} uses a background extraction model.
    \item The \textbf{estimation step} applies a constant velocity model to predict the next position for each track.
    \item The \textbf{association step} is performed applying the Hungarian algorithm on a Euclidean distance cost matrix.
    \item Finally, a \textbf{track manager} allows the creation, resuming, destruction and ending of tracks.
\end{enumerate}

\subsubsection{DA-Tracker}

{
    DA-Tracker\cite{abeysinghe2023tracking}, developed as a research project in May 2023, 
    combines detection and tracking using a model extended with domain discriminator modules. 
}

{
    This innovative approach includes an adversarial training strategy for domain adaptation alongside tracking objectives. 
    While this model was discovered during the latter stages of our project, it has been included in the state of the art section, despite not being tested.
}
