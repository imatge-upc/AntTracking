
{
    Deep SORT\cite{Wojke2017simple} is a model published in 2017, it upgrades the SORT model with a new associator, containing appearance model.
}

\paragraph{Appearance model}

{
    The key innovation in Deep SORT is the introduction of an appearance model.
    The appearance model defines an operation that transforms the image crop with 
    the detected object in the image space into a new unitary norm feature space, 
    with the resulting feature vectors represented as $r_n$.
}

%\begin{equation}
%    \label{eqn:appearance model}
%    Appearance(crop(\mathbf{I}, b_{n})) = r_{n} \; \mid \; ||\:r_{n}\:|| = 1
%\end{equation}

\paragraph{Associator}

{
    The resulting feature space enables online identity definition using classical models such as \ac{KNN}, distance to the class mean or distance to the last instance.
    In the paper, the authors use the minimum cosine distance with the last 100 instances of each track as appearance distance, it is shown in the following equation \ref{eqn:cosine distance}.
}

\begin{equation}
    \label{eqn:cosine distance}
    d_{c}(i, j) = min\{1 - r_{j}^T \cdot r_{k}^{i} \;\mid\; r_{k}^{i} \in \text{"Last 100 appearances of identity i"} \}
\end{equation}

{
    The Deep SORT associator combines two distances: the location distance, represented by the Mahalanobis distance (equation \ref{eqn:Mahalanobis distance}), and the previous appearance distance. 
    The Mahalanobis distance considers the expected covariance between detections and track estimates. 
}

\begin{equation}
    \label{eqn:Mahalanobis distance}
    d_{m}(i, j) = (z_{j} - z_{i})^T \cdot P_{i}^{-1} \cdot (z_{j} - z_{i})
\end{equation}

{
    Here, $z_j$ is the $j$-th bounding box detection, and $z_i$ is the $i$-th track's bounding box estimation. 
    $P_i$ represents the covariance matrix estimation of the $i$-th track ($\hat{\mathbf{P}}_{k}$ in equation \ref{eqn:Kalman covariance matrix estimation}).
}

{
    The associator employs these distances to make inadmissible match discards. 
    The Mahalanobis distance can be thresholded using the inverse chi-squared ($\chi^2$) distribution, 
    while the appearance distance threshold can be data-driven.
}

{
    Subsequentially, both distances are combined using a weighted sum to make a cost matrix. 
    The authors argue in favor of using only the appearance distance. 
    This cost matrix is \textbf{split by track age} before applying linear assignments within a \textbf{matching cascade}, where recently tracked tracks are prioritized.
}

{
    Finally, a last assignment is done using a \ac{IoU} cost matrix computed from the unassigned detections and unassigned tracks of age 1 (recently tracked tracks and potential new tracks from the previous frame).
}

